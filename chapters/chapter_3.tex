\newpage
\begin{center}
  \textbf{\large 3. РАСЧЕТ ВТУЛОК ФЛАНЦЕВ}
\end{center}
\refstepcounter{chapter}
\addcontentsline{toc}{chapter}{3. РАСЧЕТ ВТУЛОК ФЛАНЦЕВ}


По своей конструкции фланцы делятся на длинные цилиндрические, короткие, конические, галтельные и приварные внахлест к трубам.
Утолщение края втулки делается для снижения во фланце напряжений и прогиба его кольца.

С точки зрения расчета указанные выше втулки принято считать как тонкостенные цилиндрические оболочки, в общем случае с остенками переменной толщины; нагрузка считается приложеной к краю симметрично отностительно оси оболочки.
Все задачи, связанные с деформациями таких оболочек, своядтся к интегрированию дифференциального уравнения того же вида, что и дифференциальное уравнение балки, лежащей на упругом основании. 
Наиболее простое решение получаетсядля втулки постоянной толщины, если ее длина достаточна для того, чтобы считать равными нулю деформации в поперечном сечении достаточно удаленном от места приложения нагрузки (краевой эффект). 

Расчет конической втулки сводится к расчету балки переменного сечения, лежащей на упругом основании, с переменным коэффициентом пропорциональности между прогибом и реакцией.
Излагаемое ниже решение задачи не содержит допущений, кроме тех, которые принимаются при расчете балок на упругом основании.
Задача сводится к определгнию коэффициентов податливости края втулки, входящих в уравнение . 
Известно, что у втулок с достаточно длинной конической частью наибольшие напряжения могут возникать не в основании, а в переходе между цилиндрической иконической частью. 
В данном сечении наппяжения следует вычислять дополнительно.

Однако решение дифференциального уравнения упругой линии конической части втулки не выражается через элементарные функции, а также известно, что интегрирование такого уравнения не приводится к квадратурам.
В связи с этим, задача об изгибе конической втулки решается приближенными методами:
\begin{enumerate}
  \item Методом конечных разностей.
  \item Численным интегрированием.
  \item По методу С.П. Тимошенко, когда искомое решение представляется в виде степенного ряда.
\end{enumerate}


\section{Основные зависимости для цилиндрических втулок}
\label{CylindricalSleeve}

Как известно, дифференциальное уравнение упругой линии балки постоянного сечения, лежащей на сплошном упругом основании имеет вид
\begin{equation}
  \label{CylindricalSleeve_eq1}
  \frac{d^4 x_2}{dx_1^4} +4 {\beta}^4 x_2=0.  
\end{equation}
где 
\begin{equation}
  \label{CylindricalSleeve_eq2}
  \beta=\sqrt[4]{\frac{k}{4EJ}}.  
\end{equation}

Здесь {k} -- коэффициент податливости основания (коэффициент постели), явялется величиной постоянной; $EJ$ -- жесткость балки при изгибе.

Интегралом уравнения \eqref{CylindricalSleeve_eq1} является выражение 
\begin{equation}
  \label{CylindricalSleeve_eq3}
  x_2=e^{\beta x_1} \left( A \cos{\beta x} +B \sin {\beta x} \right)+ e^{-\beta x_1} \left( C \cos{\beta x} +D^{\prime} \sin {\beta x} \right),  
\end{equation}
где $$,$$,$$,$$ -- построянные интегрирования, определяются в каждом частном случае из граничных условий.

Коэффициент податливости основания для балки-полоски (единичной ширины), вырезанной из цилиндрической оболочки толщиной стенки $s$ и средним радиусом $r_{\text{ср}}$,
\begin{equation}
  \label{CylindricalSleeve_eq4}
  k=\frac{Es}{r_{\text{ср}}^2}.  
\end{equation}

Момент инерции $J$ поперечного сечения балки-полоски шириной 0,01 м 
\begin{equation*}
  J=\frac{s^3}{12}.  
\end{equation*}

После замены ${k}$,${J}$ в выражении \eqref{CylindricalSleeve_eq2} их значениями и учета, что выделенная из оболочки балка-полоска изгибается в условиях плоской деформации, найдем
\begin{equation}
  \label{CylindricalSleeve_eq5}
  \beta=\sqrt[4]{\frac{3 \left( 1-{\nu}^2 \right)}{r_{\text{ср}}^2 s^2}} 
\end{equation}
при $\nu=0,3$ 
\begin{equation}
  \label{CylindricalSleeve_eq6}
  \beta=\frac{1,285}{\sqrt{r_{\text{ср}} s}} 
\end{equation}

Длинная цилиндрическая втулка фланца представляет собой трубу, к которой приложены изгибающие моменты $M_0$ и поперечные силы $Q_0$, равномерно распределенной по краю $x_1=0$ рис . 
Всегда имеется сечение полоски, находящееся на конечном расстоянии от начала координат, перемещения которого равны нулю.
Из этого условия заключаем, что $A=B=0$. Уравнение \eqref{CylindricalSleeve_eq3} принимает вид
\begin{equation}
  \label{CylindricalSleeve_eq7}
  x_2= e^{-\beta x_1} \left( C \cos{\beta x_1} +D^{\prime} \sin {\beta x_1} \right),  
\end{equation}
Постоянные $C$ и $D^{\prime}$ определим из следующих граничных условий:
\begin{equation}
  \label{CylindricalSleeve_eq8}
  \begin{split}
    M_0=-D \left( \frac{d^2 x_2}{dx_1^2} \right) \bigg|_{x_1=0},\\
    Q_0=D \left( \frac{d^3 x_2}{dx_1^3} \right) \bigg|_{x_1=0}.
  \end{split} 
\end{equation}
Здесь $D$ -- цилиндрическая жесткость, определяемая по формуле , с заменой $h$ на $s$.

Подставив $x_2$ из уравнения \eqref{CylindricalSleeve_eq7} в условия \eqref{CylindricalSleeve_eq8}, получим два уравнения, содержащие $C$ и $D^{\prime}$, из которых найдем
\begin{equation*}
    C=\frac{1}{2 {\beta}^3 D} \left( Q_0 - \beta M_0 \right); \quad D^{\prime}=\frac{M_0}{2 {\beta}^2 D}
\end{equation*}

Подаставив эти занчения в уравнение \eqref{CylindricalSleeve_eq7}, получим 
\begin{equation}
  \label{CylindricalSleeve_eq9}
  x_2= \frac{e^{-\beta x_1}}{2 {\beta}^3 D} \left[ Q_0 \cos{\beta x} - \beta M_0 \left( \cos{\beta x_1}-\sin{\beta x_1} \right) \right],  
\end{equation}

Кривая, отвечающая уравнению \eqref{CylindricalSleeve_eq9} -- быстрозатухающая, с длиной волны 
\begin{equation}
  \label{CylindricalSleeve_eq10}
  \lambda=\frac{2 \pi}{\beta}=2 \pi \sqrt{\frac{r_{\text{ср}}^2 s^2}{3 \left( 1- {\nu}^2 \right)}}   
\end{equation}

Прогиб края ктулки найдем из \eqref{CylindricalSleeve_eq9}, положив $x_1=0$:
\begin{equation}
  \label{CylindricalSleeve_eq11}
  (x_2)|_{x_1=0}= -\frac{1}{2 {\beta}^3 D} \left( Q_0 - \beta M_0 \right),  
\end{equation}

Угол поворота края найдем, если продифференцируем уравнение \eqref{CylindricalSleeve_eq9} и положим $x_1=0$:
\begin{equation}
  \label{CylindricalSleeve_eq12}
  \left(\frac{d x_2}{d x_1}\right) \bigg|_{x_1=0}= -\frac{1}{2 {\beta}^2 D} \left( Q_0 -2 \beta M_0 \right),  
\end{equation}

Положив в уравнениях \eqref{CylindricalSleeve_eq11} и \eqref{CylindricalSleeve_eq12} $Q_0=1$ и $M_0=0$, затем $Q_0=0$ и $M_0=1$, найдем следующие выражения коэффициентов податливости края длинной цилиндрической втулки:
\begin{equation}
  \label{CylindricalSleeve_eq13}
  k_{11}=\frac{1}{2 {\beta}^3 D}; \quad k_{22}=\frac{1}{\beta D};\quad k_{12}=\frac{1}{2 {\beta}^2 D}.  
\end{equation}

Известно, что детерминант системы канонических уравнений никогда не может равняться нулю или быть отрицательным, значит 
\begin{equation}
  \label{CylindricalSleeve_eq14}
  k_{11}=\frac{1}{2 {\beta}^3 D}; \quad k_{22}=\frac{1}{\beta D};\quad k_{12}=\frac{1}{2 {\beta}^2 D}.  
\end{equation}

Раскрыв детерминант, находим
\begin{equation}
  \label{CylindricalSleeve_eq15}
  k_{11}k_{22} > k_{12}^2,  
\end{equation}
что после замены коэффициентов полученными их выражениями, даст 
\begin{equation}
  \label{CylindricalSleeve_eq16}
  \frac{1}{2 {\beta}^3 D} \cdot \frac{1}{\beta D} > \frac{1}{\left( 2 {\beta}^2 D \right)^2}.  
\end{equation}

Преобразовав, получим 
\begin{equation*}
  k_{11}k_{22} > 2k_{12}^2  
\end{equation*}
или 
\begin{equation}
  \label{CylindricalSleeve_eq17}
  \sqrt{k_{11}k_{22}} > 1,414k_{12}.  
\end{equation}


Коэффициенты податливости края любой длинной цилиндрической втулки удовлетворяют полученному равенству.
Заметим, что к длинным цилиндрическим втулкам принято относить втулки длиной не менее $0,5 \lambda$, к коротким -- менее $0,5 \lambda$, хотя в ряде случаев втулки длиной менее $0,5 \lambda$ можно считать длинными.
Наиболшими в длинной цилиндрической втулке являются изгибные напряжения ${\sigma}_{{\text{вт}}_2}^{\prime \prime}$ в ее основании, определяемые по формуле 
\begin{equation}
  \label{CylindricalSleeve_eq18}
  {\sigma}_{{\text{вт}}_2}^{\prime \prime}=\frac{6M_0}{s^2}.  
\end{equation}



\section{Основные зависимости для коротких цилиндрических втулок втулок}
\label{ShortConicalSleeve}

Этот случай приводится к расчету балки длиной $l_{\text{вт}}$, лежащей на упругом основании рисс. Произвольные постоянные $A$,$B$,$C$ и $D^{\prime}$ отличны от нуля, поэтому необходимо пользоваться уравнением \eqref{CylindricalSleeve_eq3}.
Граничные условия следующие: 
\begin{enumerate}
  \item при $x=0 \quad Q_x=Q_0$, $M_x=0$ или $M_x=0 \quad M_x=M_0$;
  \item при $x=l_{\text{вт}} \quad M_x=0, \quad Q_x=Q_0$.
  \end{enumerate}

Произвольные постоянные $A$,$B$,$C$ и $D^{\prime}$ можно связать с начальными параметрами $y_0$,${\varphi}_0$,$M_0$,$Q_0$, под которыми понимаем соотвественно значения прогиба, угла поворота, изгибающего момента и поперечной силы в левом сечении балки.
Затем приведем уравнение \eqref{CylindricalSleeve_eq3} к виду 
\begin{equation}
  \label{ShortConicalSleeve_eq1}
  x_2=y_0 A_x +\frac{{\varphi}_0}{\beta}B_x -\frac{4 {\beta}^2}{kb}M_0 C_x+ \frac{4 \beta}{kb}Q_0 D_x,  
\end{equation}
где $A_x$,$B_x$,$C_x$,$D_x$ есть функции Крылова:
\begin{equation}
  \label{ShortConicalSleeve_eq2}
  \begin{split}
    A_x&={\mathop{\rm ch}\nolimits}{\beta x} \cos{\beta x},\\
    B_x&=\frac{1}{2} \left( {\mathop{\rm ch}\nolimits}{\beta x}\sin{\beta x}+ {\mathop{\rm sh}\nolimits}{\beta x}\cos{\beta x}\right),\\
    C_x&={\mathop{\rm sh}\nolimits}{\beta x} \sin{\beta x},\\
    D_x&=\frac{1}{2} \left( {\mathop{\rm ch}\nolimits}{\beta x}\sin{\beta x}- {\mathop{\rm sh}\nolimits}{\beta x}\cos{\beta x}\right)
  \end{split}
  x_2=y_0 A_x +\frac{{\varphi}_0}{\beta}B_x -\frac{4 {\beta}^2}{kb}M_0 C_x+ \frac{4 \beta}{kb}Q_0 D_x,  
\end{equation}
$k$ выражено формулой \eqref{CylindricalSleeve_eq4}. 

Выражения изгибающего момента, поперечной силы и угла поворота следующие 
\begin{equation}
  \label{ShortConicalSleeve_eq3}
  \begin{split}
    M(x)&=M_0 A_x -\frac{1}{\beta}Q_0 B_x +\frac{kby_0}{{\beta}^2}C_x+\frac{kb{\varphi}_0}{{\beta}^3}D_x,\\
    Q(x)&=Q_0 A_x -\frac{kby_0}{{\beta}^2} B_x -\frac{kb{\varphi}_0}{{\beta}^3}C_x+4 \beta M_0 D_x,\\
    \varphi (x)&={\varphi}_0 A_x -\frac{k {\beta}^3 }{{kb}}M_0 B_x + \frac{k {\beta}^2 }{{kb}}Q_0 C_x - 4 \beta y_0 D_x.\\
  \end{split}
\end{equation}

Здесь и в уравнении \eqref{ShortConicalSleeve_eq1} ширину балки $b$  таже полагаем равной единице.

Из уравнений \eqref{ShortConicalSleeve_eq3} при $x=l_{\text{вт}}$, $M|_{x=l}=Q|_{x=l}=0$ можно получить выражения для начальных параметров $y_0$,${\varphi}_0$:





\section{Основные зависимости для конических втулок}
\label{ConicalSleeve}

Прогиб $y$ полубескончной балки-полоски шириной, равной единице, лежащей на упругом основании, нагруженной на левом конце моментом $M_0$ или силой $Q_0$ , удовлетворяет дифференциальному уравнению 
\begin{equation}
  \label{ConicalSleeve_eq1}
  \frac{d}{dx_1^2} \left( D \frac{d^2 x_2}{dx_1^2} \right)+kx_2=0.  
\end{equation}

При этом ось $x_1$ направлена вдоль нижней стороны недеформированной балки-полоски, $D$ -- переменная жесткость балки-полоски (цилиндрическая жесткость оболочки), $k$ -- переменный коэффициент податливости основания.
Этот коэффициент податливости определяется выражением , в котором $s$, $r_{\text{ср}}$ -- соответсвенно толщина стенки и радиус срединной поверхности есть переменные величины, функции координаты $x_1$.

Упругой линии каждого участка балки-полоски (в пределах конической и цилиндрической части втулки) соответствует свой интеграл с четырьмя постоянными интегрирования, две из которых равны нулю.
Остается шесть постоянных интегрирования.
Четыре первых краевых условия получим, приравняв перемещения и силы слева и справа то переходного сечения $CD$.
Последние два граничных условия являются условиями нагружения балки-полоски:
\begin{equation}
  \label{ConicalSleeve_eq2}
    M|_{x_1=0}=M_0 \quad Q|_{x_1=0}=Q_0 
\end{equation}

Определив постоянные интегрирования, нетрудно найти силы и перемещения в любом сечении балки-полоски.

\subsection{Расчет по методу С.П. Тимошенко}
\label{Timoshenko}

Условие тонкостенности рассматриваемой оболочки позволяет переменный радиус срединной поверхности конической части заменить постоянным радиусом срединной поверхности цилиндрической части оболочки. 
Взяв, далее, оси координат, как показано, на рис, для толщины $s_{x_1}$ и цилиндрической жесткости $D_{x_1}$ конической части, будем иметь следующие выражения:
\begin{equation}
  \label{Timoshenko_eq1}
  s_{x_1}=ax_1; \quad D_{x_1}= \frac{Ea^3}{12 \left( 1- {\nu}^2 \right)}x_1^3,
\end{equation}
где $a= \tan{\alpha/2}$ и уравнение \eqref{ConicalSleeve_eq1} примет вид 
\begin{equation}
  \label{Timoshenko_eq2}
  \frac{d}{d x_1^2} \left( x_1^3 \frac{d^2 x_2}{dx_1^2} \right)+\frac{12 \left( 1- {\nu}^2 \right)}{a^2 r_{\text{ср}}^2}x_1x_2=0.  
\end{equation}

Общее решение уравнения \eqref{Timoshenko_eq2} есть 
\begin{equation}
  \label{Timoshenko_eq3}
  y_{\text{лев}}=\frac{1}{\sqrt{x_1}} \left[ C_1 \psi_1^{\prime}(\xi)+C_2 \psi_2^{\prime}(\xi)+C_3 \psi_3^{\prime}(\xi)+C_4 \psi_4^{\prime}(\xi) \right]  
\end{equation}

Выражение превой производной функции $y$, момента $M$ и поперечной силы $Q$ в текущем состоянии конической, левой части балки-полоски следующие
\begin{equation}
  \label{Timoshenko_eq4}
  \begin{split}
    y_{\text{лев}}^{\prime} &=\frac{1}{2 x_1 \sqrt{x_1}} \left[ C_1 n(\xi)+C_2 m(\xi)+C_3 q(\xi)+C_4 p(\xi) \right],\\
    M_{\text{лев}} &=-\frac{Ea^3 \sqrt{x_1}}{48 \left( 1- {\nu}^2 \right)} \left[ C_1 d(\xi)+C_2 e(\xi)+C_3 f(\xi)+C_4 h(\xi) \right],\\
    Q_{\text{лев}} &=\frac{Ea^3 {\rho}^2 \sqrt{x_1}}{24 \left( 1- {\nu}^2 \right)} \left[ C_1 m(\xi)+C_2 n(\xi)+C_3 p(\xi)+C_4 q(\xi) \right].
  \end{split}  
\end{equation}

Здесь приняты следующие обозначения
\begin{equation}
  \label{Timoshenko_eq5}
  \begin{split}
    m(\xi)&=\xi \psi_1(\xi)+2 \psi_2^{\prime}(\xi),\\
    n(\xi)&=\xi \psi_2(\xi)-2 \psi_1^{\prime}(\xi),\\
    p(\xi)&=\xi \psi_3(\xi)+2 \psi_4^{\prime}(\xi),\\
    q(\xi)&=\xi \psi_4(\xi)-2 \psi_3^{\prime}(\xi),\\
    d(\xi)&={\xi}^2 \psi_2^{\prime}(\xi)-4 \xi \psi_2(\xi)+8 \psi_1^{\prime}(\xi),\\
    e(\xi)&={\xi}^2 \psi_1^{\prime}(\xi)-4 \xi \psi_1(\xi)-8 \psi_2^{\prime}(\xi),\\
    f(\xi)&={\xi}^2 \psi_4^{\prime}(\xi)-4 \xi \psi_4(\xi)+8 \psi_3^{\prime}(\xi),\\
    h(\xi)&={\xi}^2 \psi_3^{\prime}(\xi)-4 \xi \psi_3(\xi)-8 \psi_4^{\prime}(\xi);
  \end{split}  
\end{equation}

\begin{equation}
  \label{Timoshenko_eq6}
  \begin{split}
    {\rho}^4&=\frac{12 \left( 1- {\nu}^2 \right)}{a^2 r_{\text{ср}}^2},\\
    \xi&=2 \rho \sqrt{x_1}.
  \end{split}  
\end{equation}

Числовые значения функций вида $\psi (\xi)$ и $\psi^{\prime} (\xi)$, входящие в выражения, приведены для значений $\xi$ до 10 в таблицах, где имеют следующие значения

Последовательным дифференцированием уравнения находим угол поворота, изгибающий момент и поперечную силу для полоски, выделенной из цилиндрической части втулки 
\begin{equation}
  \label{Timoshenko_eq8}
  \begin{split}
    y^{\prime} &=\beta e^{\beta x} \left[ A \left ( \cos{\beta x} + \sin {\beta x} \right )+ B \left ( \cos{\beta x} + \sin {\beta x} \right ) \right]\\
    M &=-\frac{Es^3}{6 \left( 1- {\nu}^2 \right)}{\beta}^2 e^{\beta x} \left[ -A \sin {\beta x}+B \cos{\beta x} \right],\\
    Q &=-\frac{Es^3}{6 \left( 1- {\nu}^2 \right)}{\beta}^3 e^{\beta x} \left[ -A \left (\cos{\beta x}+\sin {\beta x}\right ) +B \left (\cos{\beta x}-\sin {\beta x}\right ) \right].
  \end{split}  
\end{equation}

Заметим, что и для правой части балки-полоски начало координат принимается согласно рис. В этом случае $x$ отрицателен и в уравнении следует положить равным нулю постоянные $C$,$D^{\prime}$.
Подставим выражение , и , в граничные условия при $x=x_1$
\begin{equation*}
  \label{Timoshenko_eq9}
  \begin{split}
    y_{\text{лев}}&=y_{\text{пр}},\\
    y_{\text{лев}}^{\prime}&=y_{\text{пр}}^{\prime},\\
    Q_{\text{лев}}&=Q_{\text{пр}},\\
    M_{\text{лев}}&=M_{\text{пр}},\\
  \end{split}  
\end{equation*}
при $x_0=x_1+l_{\text{вт}}$
\begin{equation*}
  \label{Timoshenko_eq10}
  \begin{split}
    M_{\text{лев}}&=M_0,\\
    Q_{\text{лев}}&=Q_0.\\
  \end{split}  
\end{equation*}

Полученные шесть уравнений относительно постоянных интегрирования $A$,$B$,$C_1$,$C_2$,$C_3$,$C_4$ можно записать в виде
\begin{equation}
  \label{Timoshenko_eq11}
  \begin{split}
    &C_1 \psi_1^{\prime}(\xi) +C_2 \psi_2^{\prime}(\xi)+C_3 \psi_3^{\prime}(\xi)+C_4 \psi_4^{\prime}(\xi)= \\
    & \quad =\sqrt{x_1} e^{\beta x_1} \left ( A \cos {\beta x_1}+B \sin{\beta x_1} \right ),\\
    &C_1 n_1-C_2 m_1+C_3 q_1-C_4 p_1= \\
    & \quad = 2x_1 \sqrt{x_1} \beta e^{\beta x_1} \left[ A \left ( \cos{\beta x_1} - \sin {\beta x_1} \right )+ B \left ( \cos{\beta x_1} + \sin {\beta x_1} \right ) \right],\\
    &C_1 d_1-C_2 e_1+C_3 f_1-C_4 h_1= \\
    & \quad = \frac{8 s^3 {\beta}^3 e^{\beta x_1}}{a^3 \sqrt{x_1}} \left ( -A \sin {\beta x_1}+B \cos{\beta x_1} \right ),\\
    &C_1 m_1-C_2 n_1+C_3 p_1-C_4 q_1= \\
    & \quad = -\frac{4 s^3 {\beta}^3 e^{\beta x_1}}{a^3 {\rho}^2 \sqrt{x_1}} \left[ - A \left ( \cos{\beta x_1} + \sin {\beta x_1} \right )+ B \left ( \cos{\beta x_1} - \sin {\beta x_1} \right ) \right],\\
    &M_0 =-\frac{Ea^3 \sqrt{x_0}}{48 \left( 1- {\nu}^2 \right)} \left ( C_1 d_0-C_2 e_0+C_3 f_0-C_4 h_0 \right )\\
    &Q_0 =\frac{Ea^3 {\rho}^2 \sqrt{x_0}}{24 \left( 1- {\nu}^2 \right)} \left ( C_1 m_0-C_2 n_0+C_3 p_0-C_4 q_0 \right ).
  \end{split}  
\end{equation}
где $m_1$,$n_1$,$p_1,$ $\ldots$ вычисляют для $\xi$ при $x=x_1$, а $m_0$,$n_0$,$p_0,$ $\ldots$ для $\xi$ при $x=x_1+l_{\text{вт}}$.

Уравнения \eqref{Timoshenko_eq11} решают дважды: для ${\text{I}}$ нагрузки $\left (M_0=0, Q_0=-1 \right )$, в врезультате чего находят постоянные интегрирования $C_1^{\prime}, \ldots,C_4^{\prime}$, $A^{\prime}$, $B^{\prime}$ и для ${\text{II}}$ -- $\left (M_0=1, Q_0=0 \right )$, в результате чего находят постоянные $C_1^{\prime \prime}, \ldots,C_4^{\prime \prime}$, $A^{\prime \prime}$, $B^{\prime \prime}$.

После определения постоянных интегрирования найдем коэффициенты податливости края втулки:
\begin{enumerate}
  \item При действии ${\text{I}}$ нагрузки
  \begin{equation}
    \label{Timoshenko_eq12}
    \begin{split}
      k_{21}=y({\xi}_0)&=\frac{1}{\sqrt{x_0}} \left[ C_1^{\prime} \psi_1^{\prime}({\xi}_0) +C_2^{\prime} \psi_2^{\prime}({\xi}_0)+C_3^{\prime} \psi_3^{\prime}({\xi}_0)+C_4^{\prime} \psi_4^{\prime}({\xi}_0) \right],\\
      k_{22}=y^{\prime}({\xi}_0)&=\frac{1}{2x_0 \sqrt{x_0}} \left(C_1^{\prime} n_0-C_2^{\prime} m_0+C_3^{\prime} q_0-C_4^{\prime} p_0 \right);\\
    \end{split}  
  \end{equation}
  \item При действии ${\text{II}}$ нагрузки
  \begin{equation}
    \label{Timoshenko_eq13}
    \begin{split}
      k_{11}=y({\xi}_0)&=\frac{1}{\sqrt{x_0}} \left[ C_1^{\prime \prime} \psi_1^{\prime}({\xi}_0) +C_2^{\prime \prime} \psi_2^{\prime}({\xi}_0)+C_3^{\prime \prime} \psi_3^{\prime}({\xi}_0)+C_4^{\prime \prime} \psi_4^{\prime}({\xi}_0) \right],\\
      k_{12}=y^{\prime}({\xi}_0)&=\frac{1}{2x_0 \sqrt{x_0}} \left(C_1^{\prime \prime} n_0-C_2^{\prime \prime} m_0+C_3^{\prime \prime} q_0-C_4^{\prime \prime} p_0 \right);\\
    \end{split}  
  \end{equation}
  \end{enumerate}

Подставив найденные значения постоянных интегрирования \eqref{Timoshenko_eq3}, \eqref{Timoshenko_eq4} можем определить силы в любом сечении конической втулки.





В данной главе используется моделирование молекулярной динамики для расчета фазовых диаграмм обобщенных систем Леннарда-Джонса с различными показателями притяжения.
Оценены коэффициенты диффузии и подвижности (обратной диффузии) и проанализированы спектры коллективного возбуждения на жидких бинодалиях. 
Отмечено, что зависимость коэффициента подвижности от температуры является линейной в широком диапазоне температур, а ее наклон увеличивается с увеличением показателя притяжения.
При отклонении подвижности от линейной зависимости дисперсионные соотношения коллективных возбуждений жидкости переходят от осцилирующего к монотонному виду.

\section{Роль диффузии в науке и технике}
\label{MACR-SecIntroduction}

Диффузия играет решающую роль в различных процессах переноса массы, начиная с науки и техники и заканчивая живой природой.
Диффузия выступает ключевым механизмом в биологических процессах~\cite{10.1016/j.bbagen.2013.09.037, 10.1038/s41598-018-22643-9}, а также в кинетике химических реакций.
Знание механизмов диффузии позволит добиться значительного прогресса в новых биотехнологиях и медицине, решить важные проблемы химической и фармакологической промышленности~\cite{10.1002/3527602836}.

Процесс диффузии очень хорошо изучен в газах и твердых телах.
Например, в кристаллических системах~\cite{10.1016/0079-6816(95)00039-2}, что связано с ее практической ценностью в металлургии для легирования~\cite{10.1016/s0924-0136(96)02826-9, 10.1016/j.actamat.2015.10.010, 10.1134/s1063783411110308} и эксплуатации полупроводниковой электроники~\cite{10.1103/physrevlett.84.4220, 10.1016/j.physrep.2009.10.003}.

В данной главе, используя метод молекулярной динамики (МД), моделируются обобщенные системы Леннарда-Джонса с различными показателями притяжения.
Рассчитаны температурные зависимости подвижности частиц (коэффициента обратной диффузии) на бинодали жидкость-газ. 
Здесь же рассмотрена связь между диффузией, дальнодействием межчастичного притяжения и свойствами коллективных возбуждений в простых жидкостях.

\section{Методы}
\label{MACR-SecMethods}

\subsection{Расчет фазовых диаграмм методами молекулярной динамики}
\label{MACR-SubSecMD}

В этой главе анализируются транспортные свойства и их связь с коллективными модами на жидких бинодальях.
Для систем, взаимодействующих через обобщенный потенциал Леннарда-Джонса (LJ$n$-$m$):
\begin{equation}
  U_{n-m}(r)=4 \varepsilon\left[\left(\frac{\sigma}{r}\right)^{n}-\left(\frac{\sigma}{r}\right)^{m}\right]
  \label{MACR-eq1}
\end{equation}
где $\epsilon$ и $\sigma$ -- характерные масштабы энергии и длины соответственно.
На протяжении всей статьи используются приведенные единицы измерения температуры $ T/ \epsilon \rightarrow T $, расстояния $ r/ \sigma \rightarrow r $ и плотности $ \rho \sigma ^ 3 \rightarrow n$.


Были рассмотрены потенциалы LJ$12$-$4$, LJ$12$-$5$, LJ$12$-$6$ и LJ$16$-$6$.
Чтобы сравнить полученные результаты для LJ$n$-$m$ с результатами для системы, в которой взаимодействия не являются сферически-симметричными, были также смоделировали этан~\cite{10.1021/acs.jced.6b01036}.
В выбранной модели молекула этана рассматривается как пара жестко связанных радикалов CH$_3$, взаимодействующих с радикалами других молекул через потенциал~\cite{10.1021/acs.jced.6b01036}:
\begin{equation}
  U_{\rm {ethane}}(r) = \tilde \varepsilon\left[\left(\frac{\sigma}{r}\right)^{16}-\left(\frac{\sigma}{r}\right)^{6}\right],
  \label{MACR-eq2}
\end{equation}
где $\tilde\varepsilon = 0,69396$ ккал/моль и $\sigma = 3,783$\AA.

Все МД-симуляции были выполнены в ансамбле NVT (N, V и T --количество частиц, объем системы и температура соответственно) с периодическими граничными условиями с использованием пакета моделирования \\  LAMMPS~\cite{10.1006/jcph.1995.1039}.
В первую очередь были рассчитаны линии бинодали~\cite{10.1021/jp806127j, 10.1021/jp1117213}.
Исходное состояние системы формировалось в два этапа: (i) кубическая область моделирования заполнялась равновесным кристаллом (в нашем случае ГЦК) из $N$ частиц с плотностью, соответствующей близкому к нулю давлению; 
(ii) область моделирования была расширена в направлении осей $x$ так, чтобы окончательная средняя плотность системы $\rho_a$ стала равной значениям, указанным в таблице~\ref{MACR-Table1}.
Результирующее начальное состояние показано на рис.~\ref{MACR-Figure1}(а). 
Затем температура системы линейно увеличивалась от $T_{start}$ до $T_{stop}$ в течение $n_{step}$ шагов моделирования с временным шагом $\Delta t$.
Конденсированная фаза в какой-то момент начинает испаряться, образуя сосуществование газа и конденсата, если температура ниже критической, как показано на рис.~\ref{MACR-Figure1}(b).
Принципиально то, что полученное таким образом состояние системы почти всегда имеет границы фаз, ортогональные оси $x$.
В результате плотности $\rho_g$ и $\rho_c$ газовой и конденсированной фаз соответственно могут быть рассчитаны путем подгонки профиля плотности $\rho(x)$ выражением~\cite{10.1021/jp806127j, 10.1021/jp1117213}:

\begin{equation}
  \rho(x)=\frac{\rho_{l}+\rho_{g}}{2}-\frac{\rho_{l}-\rho_{g}}{2} \tanh \left(\frac{|x|-L}{\delta}\right),
  \label{MACR-eq3}
\end{equation}
где $L$ — половина длины области моделирования, занимаемой жидкой фазой, а $\delta$ — характерная ширина границы раздела.
Пример профиля плотности системы и его аппроксимация уравнением~\eqref{MACR-eq3} показаны на рис.~\ref{MACR-Figure1}(c) гистограммой и красной линией соответственно.
Параметры моделирования для рассмотренных моделей сведены в табл.~\ref{MACR-Table1}.

\begin{table}[]
  \centering
  \begin{tabular}{|lllllcl|}
    \hline
    \multicolumn{1}{|l|}{Potential} & \multicolumn{1}{l|}{$\rho_a$} & \multicolumn{1}{l|}{$r_c$} & \multicolumn{1}{l|}{$T_{start}$} & \multicolumn{1}{l|}{$T_{stop}$} & \multicolumn{1}{l|}{$n_{step}$}                       & $\Delta t$                          \\ \hline
    \multicolumn{7}{|c|}{Значения в безразмерных единицах:}                                                                                                                                                                                                         \\ \hline
    \multicolumn{1}{|l|}{LJ12-4}    & \multicolumn{1}{l|}{0.25}     & \multicolumn{1}{l|}{15.0}  & \multicolumn{1}{l|}{1.0}         & \multicolumn{1}{l|}{5.5}        & \multicolumn{1}{c|}{\multirow{4}{*}{$3 \times 10^6$}} & \multirow{4}{*}{$5 \times 10^{-4}$} \\ \cline{1-5}
    \multicolumn{1}{|l|}{LJ12-5}    & \multicolumn{1}{l|}{0.25}     & \multicolumn{1}{l|}{10.0}  & \multicolumn{1}{l|}{0.8}         & \multicolumn{1}{l|}{2.4}        & \multicolumn{1}{c|}{}                                 &                                     \\ \cline{1-5}
    \multicolumn{1}{|l|}{LJ12-6}    & \multicolumn{1}{l|}{0.35}     & \multicolumn{1}{l|}{8.0}   & \multicolumn{1}{l|}{0.5}         & \multicolumn{1}{l|}{1.4}        & \multicolumn{1}{c|}{}                                 &                                     \\ \cline{1-5}
    \multicolumn{1}{|l|}{LJ16-6}    & \multicolumn{1}{l|}{0.31}     & \multicolumn{1}{l|}{8.0}   & \multicolumn{1}{l|}{0.8}         & \multicolumn{1}{l|}{1.6}        & \multicolumn{1}{c|}{}                                 &                                     \\ \hline
    \multicolumn{7}{|c|}{Единицы измерения СИ:}                                                                                                                                                                                                                     \\ \hline
    \multicolumn{1}{|l|}{Ethane}    & \multicolumn{1}{l|}{$0.22\mathrm{\frac{g}{cm^3}}$}     & \multicolumn{1}{l|}{$25\text{\AA}$}    & \multicolumn{1}{l|}{$80\,\mathrm{K}$}          & \multicolumn{1}{l|}{$320\,\mathrm{K}$}        & \multicolumn{1}{l|}{$2 \times 10^6$}                  & $2\,\mathrm{\text{фс}}$                                   \\ \hline
  \end{tabular}
  \caption{Параметры, используемые в МД-моделировании для бимодальных расчетов: где $\rho$ — средняя плотность системы, $r_c$ — радиус отсечки, $T_{start}$ и $T_{stop}$ — начальная и конечная температуры моделирования, соответственно, $n_{step}$ — количество шагов моделирования, а $\Delta t$ — временной шаг.}
  \label{MACR-Table1}
\end{table}


\begin{figure}[!t]
  \centering
  \includegraphics[width=150mm]{MACR-Figure1.png}
  \caption{(a) Система частиц для расчета фазовой диаграммы.
    Система частиц с потенциалом взаимодействия LJ12-6 при температуре $T=1.13$ в виде плоского слоя.
    (b) Профиль плотности системы вдоль оси $x$.
    Область с высокой плотностью представляет собой конденсат, с низкой -- газ.
    Темно-красная линия представляет собой аппроксимацию профиля плотности уравнением~\eqref{MACR-eq3}.}
  \label{MACR-Figure1}
\end{figure}


\begin{figure}[!t]
  \centering
  \includegraphics[width=150mm]{MACR-Figure2.pdf}
  \caption{Фазовая диаграмма системы LJ12-6.
    Оранжевым и синим цветами обозначены символы плотности газа и конденсата соответственно, полученные путем подгонки данных МД по уравнению~\eqref{MACR-eq3}.
    Зеленые символы -- медиана $\rho_m=(\rho_g+\rho_c)/2$.
    Сплошная красная линия соответствует уравнению~\eqref{MACR-eq4}.
    Тройные и критические точки -- синие и красные звездочки соответственно.
  }
  \label{MACR-Figure2}
\end{figure}

Вблизи критической температуры расчет плотности газа и жидкости становится затруднительным из-за усиленных флуктуаций плотности.
Однако положение критической точки на фазовой диаграмме можно вычислить, аппроксимируя жидкостную и газообразную бинодальные ветви вблизи критической точки выражением~\ref{MACR-eq4}.

В трехмерии критический индекс $\beta_c = 0,5$ для потенциала LJ$12-4$, тогда как $\beta_c = 0,325$ для LJ$12$-$5$, LJ$12$-$6$, LJ$16$-$6$ и этана, согласно предыдущим результатам~\cite{10.1021/acs.jced.6b01036,10.1021/jp9072137,10.1103/physrevlett.89.025703}.

Пример полученных бинодалей для LJ12-6 и их аппроксимации уравнением~\eqref{MACR-eq4} показаны на рис.~\ref{MACR-Figure2}.
Обратите внимание, что на конденсированной бинодали имеется явный излом (см. рис.~\ref{MACR-Figure2}), который указывает на падение плотности при плавлении и соответствует положению тройной точки.
Полученные значения $A$ и $a$ из уравнения~\eqref{MACR-eq4}, а также значения плотности и температуры критических и тройных точек для рассматриваемых систем представлены в таблице~\ref{MACR-Table2}.

\begin{figure}[!t]
  \centering
  \includegraphics[width=\linewidth]{MACR-Figure3.pdf}
  \caption{(а) Фазовые диаграммы рассматриваемых систем. 
    Фазовые диаграммы рассчитывались методом двухфазного моделирования, который описан в разделе~\ref{MACR-SecMethods}.
    Цветные точки обозначают рассчитанные бинодали, треугольники -- срединные точки.
    Сплошные серые кривые показывают диапазон температур, используемый для аппроксимации и определения параметров в уравнении ~\eqref{MACR-eq4}.
    Штриховые серые кривые соответствуют экстраполированным биноидам.
    (б) Температурная зависимость подвижности частиц.
    Подвижность частиц была рассчитана на жидких бинодальях с использованием метода, описанного в разделе~\ref{MACR-SecMethods}.
    Точки, соответствующие экстраполированным бинодалим, отмечены серым цветом. 
    Прямые линии -- линейная аппроксимация подвижности.
    На вставке показана расчетная подвижность метана.}
  \label{MACR-Figure3}
\end{figure}

\subsection{Расчет диффузии и спектров на бинодали}

Далее для расчета подвижности на конденсированной бинодали моделировались системы с плотностью и температурой, взятыми из полученных фазовых диаграмм.
Обобщенные систем Леннарда-Джонса с $N = 4,0 \times 10 ^ 3$ моделировались с шагом по времени $1,5 \times 10 ^ 5$.
Для этана использовались $N = 1,065 \times 10 ^ 4 $ молекул и проведены моделирования с шагом по времени $7.0 \times 10^5 $.
Для релаксации системы использовались первые $ 5.0 \times 10 ^ 4 $ временных шагов для обобщенных LJ-систем и $ 5.0 \times 10 ^ 5 $ для этана.
Остальные параметры были такими же, как и при расчете фазовых диаграмм.

Коэффициент самодиффузии $D$ определялся по среднеквадратичному отклонению частиц:
\begin{equation}
  \sigma^2(t) = \sum\limits_{\alpha = 1}^{N} (r_{\alpha}(t) - r_{\alpha}(0))^2 / N, \quad \sigma^2(t) = 6Dt,
  \label{MACR-eq5}
\end{equation}
где $\sigma$ — среднеквадратичное отклонение, а $t$ — время.
Подвижность $\mu$ связана с коэффициентом диффузии соотношением Эйнштейна
\begin{equation}
  \mu = \frac{D}{T},
  \label{MACR-eq6}
\end{equation}
где $T$ — температура системы.

Наконец, спектры возбуждения были получены с использованием обработки тока скорости~\cite{10.1063/1.5050708}:
\begin{equation}
  C_{L, T}(\mathbf{q}, \omega)=\int dt e^{i \omega t} \text{Re} \left\langle\mathbf{j}_{L, T}(\mathbf{q}, t) \mathbf{j}_{L, T}(-\mathbf{q}, 0)\right\rangle,
  \label{MACR-eq7}
\end{equation}
где ${\bf k}$ и $\omega$ — волновой вектор и частота,
$\mathbf{j}_{L}=\mathbf{q}(\mathbf{j} \cdot \mathbf{q} ) / q^{2}$ и $\mathbf{j}_{T}=(\mathbf{j \cdot e_{\perp})e_{\perp}}$ — продольная ($L$) и поперечная ($T$) компоненты тока частиц,\\
$\mathbf{j}(\mathbf{q}, t)=N^{-1} \sum_{s} \mathbf{v}_{s}(t) \ exp \left(i \mathbf{q} \mathbf{r}_{s}(t)\right)$ и $\mathbf{v}_{s}(t)=\dot{\mathbf{r} }_{s}(t)$ — скорость $s$-й частицы.
Суммирование ведется по всем $N$ частицам в системе.
Усреднение по каноническому ансамблю обозначается $\langle\cdots\rangle$. 
Анализ $C_{L, T}(\mathbf{q}, \omega)$ проводился с помощью методов из работы~\cite{10.1038/s41598-019-46979-y}, что позволило получить дисперсионные соотношения продольной и поперечной мод.

МД-моделирование для расчета спектров возбуждения отличается от моделирования для подвижности только длительностью временного шага. Для LJ$12$-$4$ и LJ$16$-$6$ шаг по времени был выбран как $\Delta t = 1 \times 10 ^ {-4} \sqrt {m \sigma ^ 2 / \epsilon}$, а для LJ$12$-$5$ и LJ$12$-$6$ -- $\Delta t = 5 \times 10 ^ {-4} \sqrt {m \sigma ^ 2 / \epsilon}$.

\begin{figure}[!t]
  \centering
  \includegraphics[width=160mm]{MACR-Figure4.pdf}
  \caption{(a) Температурная зависимость подвижности системы LJ$12$-$6$ вдоль жидкостной бинодали.
    Температуры, при которых рассчитывались спектры возбуждения, указаны черными стрелками.
    (b) - (f) спектры возбуждения LJ$12$-$6$ систем.
    Спектры рассчитывались путем анализа скорости течения (уравнение~\eqref{MACR-eq7}) так же, как в Ref.~\cite{10.1038/s41598-019-46979-y}.
    Красный цвет соответствует гибридным модам, серый — результатам анализа отдельных мод~\cite{10.1038/s41598-019-46979-y}.
    В левом верхнем углу указаны пониженные температуры.}
  \label{MACR-Figure4}
\end{figure}

\section{Результаты}
\label{MACR-SecResults}

Результаты расчета границ сосуществования газа и жидкости показаны на рис.~\ref{MACR-Figure3}(а).
Цветные точки обозначают бинодали, треугольники -- срединные точки.
Точки, которые использовались для аппроксимации [с использованием уравнения~(\ref{MACR-eq4})], выделены сплошной серой линией.
Экстраполированные бинодали обозначены пунктирной серой линией.
Для каждой рассматриваемой системы температура и плотность выражаются в единицах температуры и плотности тройной точки соответственно.
Последние значения вместе с параметрами критических точек приведены в Таб.~\ref{MACR-Table2}.

\begin{table}[h!]
  \centering{
    \begin{tabular}{C{1.5cm}|C{1.0cm}|C{1.0cm}|C{1.0cm}|C{1.0cm}|C{1.0cm}|C{1.0cm}}
      LJn-m & $T_{\rm CP}$ & $\rho_{\rm CP}$ & $T_{\rm TP}$ & $\rho_{\rm TP}$ & $A$ & $a$ \\ \hline
      LJ12-4 & 4.85 & 0.291 & 1.75 & 0.952 & 0.559 & 0.107 \\
      LJ12-5 & 2.18 & 0.304 & 1.03 & 0.867 & 0.804 & 0.208 \\
      LJ12-6 & 1.29 & 0.315 & 0.72 & 0.830 & 1.002 & 0.326 \\
      LJ16-6 & 1.55 & 0.316 & 0.98 & 0.816 & 0.969 & 0.334 \\
      Ethane & 305.3 & 206.7 & 90.34 & 651.9 & 113.1 & 1.158
    \end{tabular}

  }
  \caption{Значения плотностей и температур критических и тройных точек и параметры аппроксимации по уравнению~\eqref{MACR-eq4} для рассматриваемых моделей.
    Для обобщенных систем LJ температуры и плотности даны в сокращенных единицах.
    Для этана температура выражена в К, а плотность выражена в $\text{кг}/\text{м}^3$.
    Параметры критической и тройной точек для этана взяты из работы~\cite{10.1063/1.555785}.}
  \label{MACR-Table2}
\end{table}

\begin{figure}[!t]
  \centering
  \includegraphics[width=160mm]{MACR-Figure5.pdf}
  \caption{Результаты для потенциала LJ$12$-$4$.
    Рисунок аналогичен рисунку~\ref{MACR-Figure4}(a)-(f).}
  \label{MACR-Figure5}
\end{figure}


Замечено, что с увеличением дальнодействия потенциала увеличиваются как температуры тройных и критических точек, так и их отношение $T_{\rm CP}/T_{\rm TP}$.

Затем по рассчитанным фазовым диаграммам была вычислена подвижность частиц при плотностях и температурах, соответствующих бинодали жидкости.
Полученная зависимость подвижности частиц от температуры представлена на рис.~\ref{MACR-Figure3}(b).
Цветные точки на (b) соответствуют цветным точкам на (a).
Серые точки обозначают подвижности на экстраполированных частях бинодали.

\begin{figure}[!t]
  \centering
  \includegraphics[width=160mm]{MACR-Figure6.pdf}
  \caption{Результаты для потенциала LJ$12$-$5$.
    Рисунок аналогичен рисунку~\ref{MACR-Figure4}(a)-(f).}
  \label{MACR-Figure6}
\end{figure}

Отметим, что при низких температурах подвижность на бинодали имеет линейную зависимость от температуры.
Ее наклон увеличивается с уменьшением дальнодействующего характера потенциала взаимодействия (т.е. с увеличением показателя притяжения).
Линейная зависимость сохраняется до определенной температуры, после которой происходит отклонение от линейной зависимости.
Возникновение такой нелинейности может быть связано с особенностями коллективной динамики частиц, которые должны коррелировать со спектрами коллективных возбуждений.

Вычисленные спектры системы LJ$12$-$6$ показаны на рис.~\ref{MACR-Figure4}.
На рисунке~\ref{MACR-Figure4}(а) изображены зависимости подвижности от температуры, а черными стрелками указаны температуры, при которых рассчитывались спектры. 
Были выбраны точки вблизи температуры, при которой наблюдается начало нелинейной зависимости, а также температура вблизи тройной точки.
На рисунке~\ref{MACR-Figure4}(b)-(f) показаны вычисленные дисперсии продольных и поперечных мод в этих точках.
Красный цвет соответствует модели с двумя осцилляторами, а серый — одномодовому анализу~\cite{10.1038/s41598-019-46979-y}.

Нетрудно заметить, что по мере приближения температуры к точке, соответствующей возникновению нелинейной зависимости, дисперсионные соотношения демонстрируют переход от осциллирующей к монотонной зависимости от волнового числа.
Таким образом, качественное изменение температурной зависимости подвижности частиц сопровождается изменением спектров возбуждения.
Наблюдаемая картина не является особенной для системы LJ$12$-$6$.
Аналогичная тенденция замечена и в других исследованных обобщенных ЛД-систем.
Это наблюдение дает новое свидетельство тесной связи между диффузией и свойствами коллективных возбуждений.


\begin{figure}[!t]
  \centering
  \includegraphics[width=160mm]{MACR-Figure7.pdf}
  \caption{Результаты для потенциала LJ$16$-$6$.
    Рисунок аналогичен рисунку~\ref{MACR-Figure4}(a)-(f).}
  \label{MACR-Figure7}
\end{figure}


\section{Заключение главы}
\label{MACR-SecConclusions}

В данной главе исследовано влияние формы потенциала парного взаимодействия на фазовые диаграммы и подвижность частиц в жидкой фазе.
Были рассчитаны кривые сосуществования газа и жидкости для потенциалов с переменным дальнодействием притяжения.
Отмечено, что с увеличением дальнодействующего характера потенциала температуры тройной и критической точек, а также их отношение $T_{\rm CP}/T_{\rm TP}$ увеличиваются.
Коэффициент диффузии и обратный ему коэффициент подвижности вычислялись на жидких бинодалиях.
Было обнаружено, что температурная зависимость подвижности линейна в широком диапазоне температур с тем большим наклоном, чем меньше диапазон притяжения.
Кроме того, установлено, что начало нелинейной температурной зависимости подвижности при высоких температурах совпадает с переходом дисперсионных зависимостей коллективных возбуждений от осциллирующей к монотонной зависимости от волнового числа.
Полученные результаты дают возможность для дальнейшего изучения диффузии и ее связи с коллективными процессами в конденсированных многочастичных системах.
